\chapter[Introdução]{Introdução}

\section{Contextualização}
\section{Objetivo}
\subsection{Geral}
\begin{itemize}
  \item Explorar e validar definições e monitoramento de produtvidade em times ágeis no mercado atual de Egenharia de Software
\end{itemize}
\subsection{Específicos}
\begin{itemize}
  \item Levantar quais as definições que a literatura traz referente a Produtividade no contexto de Engenharia de Software.
  \item Levantar quais métricas a literatura traz no contexto do monitoramento da produtividade de times ágeis.
  \item Levantar quais métricas a literatura dizer ser efetiva no monitotamento da produtividade de times ágeis.
  \item Validar se a literatura  está de acordo com o mercado de Engenharia de Software atualmente.
  \item Levantar quais definições e métricas são de fato utilizadas no atual mercado de Engenharia de Software.
\end{itemize}

\section{Organização do Documento}

  Este trabalho está organizado em quatro capítulos, incluindo este capítulo de introdução, que compreende a contextualização e objetivos do presente trabalho.

  O Capítulo 2 abrange o referencial teórico em que se baseia a pesquisa. Contém a Pesquisa Bibliográfica com definições necessárias sobre Metodologias Ágeis
  e Produtividade.  Se tratando de Produtvidade o capítulo também traz conceitos importantes referente a sua definição, fatores que o influenciam e métricas.

  No Capítulo 3 tem-se o detalhamento das metodologias utilizadas para a composião deste trabalho. A Revisão Sistemática de Literatura onde foi levantado o entendimento
  da literatura em relação a Produtividade em Times Ágeis e o Survey que tem como obejtivo validar se o que a literatura traz está de acordo com o mercado.

  O Capítulo 4 mostra a hipotese levantada e o survey a ser utilizado para a validação da mesma.

  O Capítulo 5 traz a análise de resultados provenientes do survey aplicado.

  O Capítulo 6 contém as considerações finais e trabalhos futuros.
