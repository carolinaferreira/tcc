\chapter[Metodologia]{Metodologia}
  Este capítulo tem como objetivo a metologia assim como os fluxos que serão
  seguidos a fim de atingir os objetivos da pesquisa.

  Para a coleta de dados foram utilizadas técnicas de pesquisa bibliográfica
  contendo uma revisão sistemática de literatura, que permitiu uma pesquisa sistemática
  sobre a assunto deste trabalho.

  O plano metodológico compreendeu 06 fases: Planejamento da Pesquisa; Coleta de Dados; Formulação da hipótese;
  Elaboração do Survey; Resultados; e Validação, como apresentado na figura.

  TEM UMA IMAGEM AQUI

\section{Detalhamento do plano metodológico}
\subsection{Planejamento da Pesquisa}
  Fase essencial para levantamento do problema a ser tratado na pesquisa, foram realizadas pesquisas sobre
  as métricas de utilizidas em times ágeis e onde se encontravam suas deficiências. A partir daí notou-se a falta de infromações
  relativas a produtividade dos times, o que se tornou objeto de estudo desta pesquisa.
\subsection{Coleta de Dados}
\subsubsection{Pesquisa Bibilográfica}
  A partir de pesquisas em bases de dados científicas foram levantadas informações sobre definições,
  fatores e métricas de produtividade. No Capítulo 2 é apresentado um levantamento sobre os itens citados anterioremente.
\subsubsubsection{Revisão Sistemática}
Uma revisão sistemática, assim como outros tipos de estudo de revisão, é uma
forma de pesquisa que utiliza como fonte de dados a literatura sobre determinado
tema. Esse tipo de investigação disponibiliza um resumo das evidências
relacionadas a uma estratégia de intervenção específica, mediante a aplicação
de métodos explícitos e sistematizados de busca, apreciação crítica e síntese
da informação selecionada \cite{SAMPAIO2007}.

De forma simples: "Uma revisão sistemática (literatura) é um meio de avaliar e interpretar todas as pesquisas
disponíveis relevantes para uma questão de pesquisa em particular ou área temática ou fenômeno de
interesse" \cite{Kitchenham04}.

Os passos necessários para a execução da técnica de revisão sistemática foram
propostos \cite{Brereton:2007:LAS:1225950.1226109}. As atividades propostas no modelo podem ser divididas em três fases:
planejamento, condução e documentação. Como ilustrado na figura abaixo.

TEM UMA IMAGEM AQUI

Na primeira fase tem-se como objetivo especificar questões de pesquisa a serem respondidas,
definir protocolo de pesquisa a ser seguido e validar o mesmo. Na fase de condução as atividades de
identificação, seleção, avaliação, extração e sintetização dos dados e estudos relevantes são feitas.
Na terceira e última fase temos o relato da revisão e sua validação.

\subsubsection{Survey}
Uma survey não é apenas o instrumento (o questionário ou lista de verificação)
para coletar informações. É um abrangente método de pesquisa para coletar
informações para descrever, comparar ou explicar conhecimentos, atitudes e comportamento \cite{fink}.

A pesquisa de Survey é, segundo \cite{babbie}, bastante semelhante ao tipo
de pesquisa de “censo”, onde o que diferencia as duas pesquisas é que o
“survey examina uma amostra da população, enquanto o censo geralmente implica
uma enumeração da população toda.”.

O objetivo do seu uso no presente trabalho é levantar como está o mercado de
Engenharia de Software hoje, quando se trata de Produtividade de Times Ágeis.

\section{Revisão Sistemática de Literatura}
\subsection{Planejamento}
\subsubsection{Objetivos}
O objetivo deste trabalho de revisão sistemática foi levantar informações relativas
a definição e métricas de Produtividade no contexto de
Engenharia de Software.

Seguindo o que foi proposto por \cite{Brereton:2007:LAS:1225950.1226109} na fase
de planejamento e considerando o objetivo definido, três questões de pesquisa,
que nortearam a revisão, foram formuladas:

\begin{itemize}
  \item Q1. Quais as definições de Produtvidade no contexto de Engenharia de Software?
  \item Q2. Quais métricas são utilizadas para monitorar a produtividade de times ágeis?
  \item Q3. Quais métricas são efetivas para o monitoramento?
\end{itemize}

\subsubsubsection{Planejamento da Pesquisa}
Nessa subseção estão descritos detalhes sobre a pesquisa realizada, como palavras-chave
utilizadas na construção da string de busca e as bases de dados utilizadas.

O inglês foi escolhido como idioma primário para condução da busca nas bases de dados,
para ter uma percepção no contexto global, visto que grande parte das publicações relacionada ao
tema de pesquisa são feitas nesse idioma.

Levando em consideração as questões de pesquisa levantadas após definição do objetivo do
trabalho, palavras-chave que auxiliassem na construção da string de busca foram identificadas, de
forma contemplassem o tópico de pesquisa abordado por essa revisão sistemática.

As palavras-chave identificadas foram:

\begin{itemize}
  \item PI1. Agile;
  \item PI2. Metrics;
  \item PI3. Teams;
  \item PI4. Measur;
  \item PI5. Agile Project Management;
  \item PI5. Software Productivity;
  \item PI6. Productivity Factors;
  \item PI7. Team performance;
  \item PI8. Agile Software Development.
\end{itemize}

As bases eletrônicas científicas de dados utilizadas na condução da revisão sistemática
foram selecionadas de acordo com a abrangência e disponibilidade de acesso:

\begin{itemize}
  \item BD1. IEEE Xplore;
  \item BD2. ACM Library;
  \item BD3. Springer Link;
  \item BD4. Science Direct;
\end{itemize}

A string de busca foi refinada 4 vezes para melhor se ajustar ao tema de
pesquisa, no quadro abaixo é possível observar tal evolução:

\begin{itemize}
  \item SB1. “agile” and “metrics” and “teams”
  \item SB2. ("agile software development") AND (measur* OR metric*) AND ("team")
  \item SB3. ("agile software development" OR "agile project management" )
  AND (measur* OR metric*) AND ("teams")
  \item SB4. Software productivity AND productivity factors AND team performance
  AND agile software development \\
\end{itemize}

A string SB4 foi ajustada para ser utilizada na BD3 E BD4:
\begin{itemize}
  \item SB5. "Software productivity" AND "productivity factors" AND "agile
  software development"
\end{itemize}

Logo, foram utilizadas as string SB4 e SB5 na presente revisão sistemática de
literatura.

\subsubsubsection{Critérios de Seleção}
Para selecionar o artigos a serem utilizados neste trabalho, foram avaliados
conforme critérios de inclusão e exclusão.

Critérios de inclusão:
\begin{itemize}
  \item CI1. A publicação deve estar escrita em inglês ou português.
  \item CI2. A publicação deve estar disponível para download na íntegra ou para
  leitura gratuita online.
  \item CI3. A publicação deve apresentar estudos relevantes ao tema proposto
  nesta revisão sistemática no título ou no resumo.
\end{itemize}

Critérios de exclusão:
\begin{itemize}
  \item CE1. A publicação que se encontrar duplicada na outra base de dados
  pesquisada.
  \item CE2. A publicação que não entra nos critérios de inclusão.
\end{itemize}

\subsubsection{Execução}

Foram iniciados os procedimentos para executar a revisão, após o planejamento da
revisão sistemática, seleção das fontes de busca, definição da string de busca,
elicitação dos critérios de inclusão e exclusão dos resultados. As publicações
encontradas a partir da busca nas bases de dados foram documentadas.

Foram encontrados um total de 27 artigos, sendo 16 obtidos através da string de
busca SB4 e 11 pela string de busca SB5.

\begin{table}[h]
	\centering
	\caption{Quantidade de obras}
	\label{tab01}

	\begin{tabular}{ccc}
		\toprule
		\textbf{Base de Dados} & \textbf{Quantidade}\\
		\midrule
		IEEE Xplore & 6 \\
		ACM Library & 10 \\
		Springer Link & 8 \\
		Science Direct 4 & 3 \\
		\bottomrule
    \textbf{Total} & \textbf{27} \\
	\end{tabular}
\end{table}

Após aplicar os critério de seleção, 21 publicações foram eliminadas, como
descrito na tabela abaixo:

\begin{table}[h]
	\centering
	\caption{Quantidade de obras eliminadas}
	\label{tab01}

	\begin{tabular}{ccc}
		\toprule
		\textbf{Base de Dados} & \textbf{Quantidade}\\
		\midrule
		IEEE Xplore & 6 \\
		ACM Library & 8 \\
		Springer Link & 6 \\
		Science Direct 4 & 1 \\
		\bottomrule
    \textbf{Total} & \textbf{21} \\
	\end{tabular}
\end{table}

O quadro abaixo ilustra as obras utilizadas, de acordo com as questões de pesquisa.
A primeira coluna contendo a classificação por grupo dos artigos, a segunda contém a
numeração dos trabalhos, a terceira contém os títulos dos trabalhos lidos por completo,
a quarta contém o nome dos autores e a quinta contém o ano de publicação.

\subsubsection{Resultados}
\subsubsection{Considerações Finais}
\section{Survey}
